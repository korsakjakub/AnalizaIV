\documentclass[a5paper, oneside]{memoir}
\usepackage[utf8]{inputenc}
\usepackage[margin=0.5in]{geometry}
\addtolength{\topmargin}{0.2in}
\addtolength{\textheight}{-.25in}
\usepackage{mdframed}
\usepackage{xcolor}
\usepackage{subfiles}
\usepackage{subcaption}
\usepackage{enumerate}
\usepackage{graphicx}
\usepackage{wrapfig}
\usepackage{amsmath}
\usepackage{amsfonts}
\usepackage{amssymb}
\usepackage{amsthm}
\usepackage{polski}
\usepackage{import}
\usepackage{xifthen}
\usepackage{pdfpages}
\usepackage{transparent}
\usepackage{fancyhdr}
\usepackage{lastpage}
\usepackage{titlesec}
\usepackage{hyperref}
\hypersetup{
    linktoc=all,     %set to all if you want both sections and subsections linked
    linkcolor=blue,  %choose some color if you want links to stand out
}

\pagestyle{fancy}
\fancyhead{}
\fancyfoot{}
\fancyhead[RO, LE]{\thepage}
\fancyhead[C]{Analiza IV}
\let\LaTeXStandardTableOfContents\tableofcontents

\renewcommand{\tableofcontents}{%
\begingroup%
\renewcommand{\bfseries}{\relax}%
\LaTeXStandardTableOfContents%
\endgroup%
}%

\chapterstyle{dash}


\titleformat{\chapter}[block]
  {\normalfont\LARGE}{\thechapter.}{1em}{\Large}
\titlespacing*{\chapter}{0pt}{-19pt}{32pt}

\newmdtheoremenv{tw}{Twierdzenie}
\newmdtheoremenv{stw}{Stwierdzenie}
\newmdtheoremenv{definicja}{Definicja}
\newtheorem{pytanie}{Pytanie}
\newtheorem{przyklad}{Przykład}
\newtheorem{uwaga}{Uwaga}

\DeclareMathOperator{\ilwewn}{\lrcorner\,}
\DeclareMathOperator{\supp}{supp}

\DeclareMathOperator{\Res}{Res}
\DeclareMathOperator{\sinc}{sinc}
\graphicspath{
    {"../img/"}
    {"img/"}
}

\title{\Huge \textbf{{\tiny\textit{korona}}Wykłady z Analizy IV}}
\author{Jakub Korsak}
\date{II 2020 - VI 2020}
\begin{document}
\frontmatter
\maketitle
\pagebreak
\tableofcontents
\mainmatter
\chapter{25.02.2020, \textbf{Wykład }\textit{O zbieżności szeregów Fouriera i próbach unikania schodków}}
\subfile{tex/w1.tex}
\chapter{28.02.2020, \textbf{Ćwiczenia }\textit{Metoda Frobeniusa}}
\subfile{tex/c1.tex}
\chapter{03.03.2020, \textbf{Wykład }\textit{Wstęp do równań cząstkowych, metoda charakterystyk}}
\subfile{tex/w2.tex}
\chapter{??.03.2020, \textbf{Ćwiczenia }\textit{Metoda charakterystyk}}
\subfile{tex/c2.tex}
\chapter{10.03.2020, \textbf{Wykład }\textit{Metoda charakterystyk}}
\subfile{tex/w3.tex}
\chapter{\textbf{Wykład }\textit{Metoda separacji zmiennych oraz wstęp do operatorów Sturma-Liouville'a}}
\subfile{tex/w5.tex}
\chapter{\textbf{Wykład }\textit{Operatory Sturma-Liouville'a}}
\subfile{tex/w6.tex}
\chapter{\textbf{Wykład }\textit{Czy można uporządkować poziomy energetyczne?}}
\subfile{tex/w7.tex}
\chapter{\textbf{Wykład }\textit{Równanie falowe}}
\subfile{tex/w8.tex}
\chapter{\textbf{Wykład }\textit{Droga do wzoru Kirchoffa}}
\subfile{tex/w9.tex}
\chapter{\textbf{Wykład }\textit{Czy z Kirchoffa w 3D można wyprowadzić 2D i 1D? Co to znaczy nadawanie morsem?}}
\subfile{tex/w10.tex}
\chapter{\textbf{Wykład }\textit{Całki Duhamela, jednoznaczność równania falowego}}
\subfile{tex/w11.tex}
\chapter{\textbf{Wykład }\textit{Jednoznaczność równania przewodnictwa}}
\subfile{tex/w12.tex}
\chapter{\textbf{Wykład }\textit{Przykłady (nie)fajności metody separacji zmiennych}}
\subfile{tex/w13.tex}
\chapter{\textbf{Wykład }\textit{Rozwiązywanie przez rozwijanie w szereg a później sprowadzenie na Ziemię przykładem Kowalewskiej}}
\subfile{tex/w14.tex}


\end{document}
